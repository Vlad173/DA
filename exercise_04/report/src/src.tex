\section{Описание}
Алгоритм Ахо — Корасик — алгоритм поиска подстроки, разработанный Альфредом Ахо и Маргарет Корасик в 1975 году, 
реализует поиск множества подстрок из словаря в данной строке. 
Широко применяется в системном программном обеспечении, например, используется в утилите поиска grep.


Описание кода: Считывается первая строка входных данных - образец, делится по джокерам на подобразцы, и составляется из
 них trie. 
В каждой вершине структуры содержатся: ссылки перехода в формате unordered\_map<Значение, ссылка на следующую вершину>, 
суфиксная 
ссылка, ссылка выхода, является ли вершина концом слова, позиции в главном образце, оканчивающихся в этой вершине 
подобразцов, длина подобразца.
После разбора всего образца, происходит <<прошивка>>  структуры, сначала <<прошиваются>> корень и следующие за ним 
вершины,
затем следующие уровни по очереди.
Далее считывается весь текст в вектор <символ, <строка, позиция>> и отправляется на поиск.
В котором создается вектор равный размеру текста и заполняется нулями, и при нахождении какого-либо образца в этом 
веторе элемент под индексом [текущая позиция в тексте - позиция подобразца в образце] увеличивался на единицу.
В итоге образец считается найденным в позиции i, если в масиве на позиции i, число равно количеству образцов.
\pagebreak

\section{Исходный код}

\begin{longtable}{|p{10.5cm}|p{4.5cm}|}
	\hline
	\rowcolor{lightgray}
	\multicolumn{2}{|c|} {trie.cpp}\\
	\hline
	TTrie::TTrie() & Конструктор по умол.\\
	\hline
	void TTrie::add(std::pair<std::vector<std::string>, size\_t>\& pat) & Добавление шаблона в trie\\
	\hline
	void TTrie::process(size\_t max) & Прошивка дерева\\
	\hline
	void TTrie::deleteTrie(TNode *node) & Очистка памяти\\
	\hline
	std::vector<std::pair<size\_t, size\_t>> TTrie::search(std::vector<std::pair<std::string, std::pair<size\_t, size\_t>>>\& text, size\_t patCount, size\_t patLen) & Алгоритм поиска подстроки\\
	\hline
	TTrie::~TTrie() & Деструктор\\
	\hline
\end{longtable}

\begin{lstlisting}[language=C]
	#pragma once
	#include <iostream>
	#include <string>
	#include <unordered_map>
	#include <utility>
	#include <vector>
	class TTrie;
	class TNode {
	public:
		friend TTrie;
		TNode() : failLink(nullptr), exitLink(nullptr), leaf(false) {}
	private:
		std::unordered_map<std::string, TNode*> childs;
		TNode *failLink, *exitLink;
		bool leaf;
		size_t patSize;
		std::vector<size_t> pos;
	};
	class TTrie {
	public:
		TTrie();
		~TTrie();
		void add(std::pair<std::vector<std::string>, size_t> &pat);
		void process(size_t max);
		std::vector<std::pair<size_t, size_t>> search(std::vector<std::pair<std::string, std::pair<size_t, size_t>>>& text, size_t patCount, size_t patLen);
	private:
		void deleteTrie(TNode* node);
		void processLvl(TNode *node, size_t max);
		void processNode(TNode *parent, TNode *node, std::string nodeSym);
		TNode* root;
	};
\end{lstlisting}

\section{Консоль}
\begin{alltt}
MacBook:solution vladislove$ ./solution 
cat dog cat dog bird
CAT dog CaT Dog Cat
DOG bird CAT
dog cat dog bird
1, 3
2, 3

MacBook:solution vladislove$ ./solution 
a b ? c d e ? ? a b
a b c d e 
a b c c d
e 
a
b a b
2, 1

MacBook:solution vladislove$ ./solution 
? ? ? ? ?       
MAI eto ya
MAI eto mi
MAI eto lucshie ludi strani
1, 1
1, 2
1, 3
2, 1
2, 2
2, 3
3, 1
\end{alltt}
\pagebreak

