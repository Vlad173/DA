\section{Описание}
Требуется написать реализацию AVL-дерева. AVL-дерево - дерево, каждый узел которого соответствует условию: модуль разности высоты 
левого поддерева и правого поддерева <= 1. Чтобы поддерживать данное условие, нужно уметь считать баланс каждого узла, 
баланс равен разности высоты левого и правого поддерева. 
При вставке и удалении требуется учитывать условие AVL-дерева и при нарушении делать перебалансировку дерева при помощи поворотов.

\pagebreak

\section{Исходный код}

\begin{longtable}{|p{10.5cm}|p{4.5cm}|}
	\hline
	\rowcolor{lightgray}
	\multicolumn{2}{|c|} {avl.cpp}\\
	\hline
	TAvlNode::TAvlNode() & Конструктор по умол.\\
	\hline
	TAvlNode::TAvlNode(char* k, uint64\_t val, int h) & Конструктор с параметрами (ключ, значение, высота)\\
	\hline
	TAvlTree::TAvlTree() & Конструктор по умолчанию\\
	\hline
	TAvlTree::$\sim$TAvlTree() & Деструктор (удаление дерева)\\
	\hline
	int TAvlTree::GetHeight(const TAvlNode* node) const & Получение высоты из узла\\
	\hline
	int TAvlTree::GetBalance(const TAvlNode* node) const & Вычисление баланса узла\\
	\hline
	void TAvlTree::RecountHeight(TAvlNode* node) & Пересчитывание высоты после поворотов\\
	\hline
	TAvlNode* TAvlTree::RightRotate(TAvlNode* node)
	TAvlNode* TAvlTree::LeftRotate(TAvlNode* node)
	TAvlNode* TAvlTree::DoubleRightRotate(TAvlNode* node)
	TAvlNode* TAvlTree::DoubleLeftRotate(TAvlNode* node)
	 & Малые и большие повороты относительно узла\\
	\hline
	TAvlNode* TAvlTree::Balance(TAvlNode* node) & Балансировка дерева\\
	\hline
	TAvlNode* TAvlTree::SubInsert(TAvlNode* node, char* key, uint64\_t value) & Вставка узла\\
	\hline
	void TAvlTree::Insert(char* key, uint64\_t value) & Обёртка для вставки\\
	\hline
	TAvlNode* TAvlTree::Find(const char* key) const & Поиск элемента\\
	\hline
	AvlNode* TAvlTree::RemoveMin(TAvlNode* node, TAvlNode* curr) & Удаление минимума в поддереве\\
	\hline
	TAvlNode* TAvlTree::SubRemove(TAvlNode* node, const char* key) & Удаление узла \\
	\hline
	void TAvlTree::Remove(const char* key) & Обёртка для удаления\\
	\hline
	void TAvlTree::SubDeleteTree(TAvlNode* node) & Рекурсивное удаление дерева\\
	\hline
	void TAvlTree::DeleteTree() & Обёртка для удалениея дерева\\
	\hline
	void TAvlTree::SubSave(std::ostream\& os, const TAvlNode* node) & Сохранение дерева в бинарный файл\\
	\hline
	TAvlNode* TAvlTree::SubLoad(std::istream\& is) & Загрузка дерева из файла\\
	\hline
	\end{longtable}

\begin{lstlisting}[language=C]
#pragma once
#include <iostream>
#include <stdint.h>
#include <cstring>

#define KEY_SIZE 257

struct TAvlNode {
	char key[KEY_SIZE] = {'\0'};
	uint64_t value;
	int height;
	TAvlNode* left;
	TAvlNode* right;

	TAvlNode();
	TAvlNode(char k[KEY_SIZE], uint64_t val, int h = 1);
	~TAvlNode();
};

class TAvlTree {
private:
	TAvlNode* root;

	int GetHeight(const TAvlNode* node) const;
	int GetBalance(const TAvlNode* node) const;
	TAvlNode* RightRotate(TAvlNode* node);
	TAvlNode* LeftRotate(TAvlNode* node);
	TAvlNode* DoubleRightRotate(TAvlNode* node);
	TAvlNode* DoubleLeftRotate(TAvlNode* node);
	TAvlNode* Balance(TAvlNode* node);
	void RecountHeight(TAvlNode* node);
	TAvlNode* SubInsert(TAvlNode* node, char key[KEY_SIZE], uint64_t value);
	TAvlNode* RemoveMin(TAvlNode* node, TAvlNode* curr);
	TAvlNode* SubRemove(TAvlNode* node, const char key[KEY_SIZE]);
	void SubDeleteTree(TAvlNode* node);
	void SubSave(std::ostream& os, const TAvlNode* node);
	TAvlNode* SubLoad(std::istream& is);

public:
	TAvlTree();
	~TAvlTree();

	TAvlNode* Find(const char key[KEY_SIZE]) const;
	void Insert(char key[KEY_SIZE], u_int64_t value);
	void Remove(const char key[KEY_SIZE]);
	void DeleteTree();
	void Save(const char path[KEY_SIZE]);
	void Load(const char path[KEY_SIZE]);
};
	
\end{lstlisting}

\section{Консоль}
\begin{alltt}
+ word 34
OK
word
OK: 34
- word
OK
word
NoSuchWord
+ till 35
OK
+ till 67
Exist
! Save tree.bin
OK
till
OK: 35
- till
OK
till
NoSuchWord
! Load tree.bin
OK
till
OK: 35
- till
OK
till
NoSuchWord
\end{alltt}
\pagebreak

