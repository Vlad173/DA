\section{Выводы}
Выполнив вторую лабораторную работу по курсу \enquote{Дискретный анализ}, я познакомился с различными структурами данных.
Научился работать с AVL-деревом. Такие структуры данных хорошо подходят для хранения и обработки большого объёма данных, так как поиск,
вставка и удаление делаются за O(log(n)). Также важно знать, как устроены эти структуры, чтобы понимать, как работают
некоторые стандартные контейнеры, например, std::map использует внутри RB-дерево.
\pagebreak
