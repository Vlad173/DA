\section{Описание}
Требуется написать реализацию алгоритма cортировки подсчётом. В качестве ключа выступают почтовые индексы.

Как сказано в \cite{wikipedia_itmo_sort}: \enquote{Идея алгоритма состоит в предварительном подсчете количества элементов с различными ключами в исходном массиве и разделении результирующего массива на части соответствующей длины (будем называть их блоками). Затем при повторном проходе исходного массива каждый его элемент копируется в специально отведенный его ключу блок, в первую свободную ячейку. Это осуществляется с помощью массива индексов $P$, в котором хранятся индексы начала блоков для различных ключей. $P[key]$ — индекс в результирующем массиве, соответствующий первому элементу блока для ключа $key$.}.

Алгоритм состоит из двух проходов по массиву $A$ размера $n$ и одного прохода по массиву $P$ размера $k$. Его трудоемкость, таким образом, равна $O(n+k)$. На практике сортировку подсчетом имеет смысл применять, если $k=O(n)$, поэтому можно считать время работы алгоритма равным $O(n)$.
Как и в обычной сортировке подсчетом, требуется $O(n+k)$ дополнительной памяти — на хранение массива B размера n и массива $P$ размера $k$.

\pagebreak

\section{Исходный код}
Разобьем процесс написания программы на несколько этапов

\begin{enumerate}
    \item Реализация необходимых новых типов (пара \enquote{ключ-значение} и вектор)
    \item Реализация алгоритма сортировки подсчетом вектора пар по ключу по заданному разряду заданной длины
    \item Реализация ввода-вывода
    \item Реализация бенчмарка
\end{enumerate}

Так как все функции и типы шаблонные, то пара, вектор и сортировки будут находиться в заголовочных файлах $pair.hpp$, $vector.hpp$ и $sort.hpp$ соответственно.
Начнем с реализации пары и вектора:

Структура $TPair$ для хранения пар \enquote{ключ-значение}:
\begin{lstlisting}[language=C++]
    #pragma once

    #include <iostream>
    
    struct TPair {
        uint32_t key;
        char value[65];
    };
    
    std::ostream& operator<<(std::ostream& output, const TPair& p) {
        output.fill('0');
        output.width(6);
        output << p.key << '\t' << p.value;
        return output;
    }
    
    std::istream& operator>>(std::istream& input, TPair& p) {
        input >> p.key >> p.value;
        return input;
    }
\end{lstlisting}

Шаблонный класс $TVector<T>$ для хранения пар \enquote{ключ-значение}:

\begin{longtable}{|p{7.5cm}|p{7.5cm}|}
\hline
\rowcolor{lightgray}
\multicolumn{2}{|c|} {vector.h}\\\hline
TVector()&Конструктор по умолчанию\\\hline
TVector(size\_t size, T\& value)&Конструктор от двух аргументов:\\ 
&размер вектора и значения по умолчанию\\\hline
TVector(const TVector\& other)&Конструктор копирования\\ 
: TVector();&\\\hline
TVector\& operator=(TVector other)&Оператор присваивания с копированием\\\hline
~TVector()&Деструктор\\\hline
size\_t Size() const&Размер вектора\\\hline
void Swap(TVector\& lhs, TVector\& rhs)&Обмен векторов значениями\\\hline
void PushBack(const T\& value)&Добавление элемента в конец\\\hline
const T\& operator[](size\_t index) const&Получение константной ссылки на элемент по индексу\\\hline
T\& operator[](size\_t index)&Получение ссылки на элемент по индексу\\\hline
\end{longtable}

\begin{lstlisting}[language=C++]
    template <typename T> 
    class TVector {
    public:
        TVector() {}
        TVector(size_t size) {}
        TVector(size_t size, uint64_t value) {}
        TVector(const TVector& other) : TVector() {}
        
        size_t Size() const {}
    
        void PushBack(const T& value) {}
    
        const T& operator[](size_t index) const {}
    
        T& operator[](size_t index) {}
    
        void Swap(TVector& lhs, TVector& rhs) {}
    
        TVector& operator=(TVector other) {}
    
        ~TVector() {}
    
    private:
        size_t size_;
        size_t capacity_;
        T* body_;
    
        void Swap(size_t& lhs, size_t& rhs) {}
    };
};
\end{lstlisting}

\newpage Теперь перейдём к реализации алгоритма сортировки подсчетом по ключу:

\begin{lstlisting}[language=C++]
    void CountingSort(TVector<TPair>& elems) {
        uint32_t max = 0;
        for (size_t i = 0; i < elems.Size(); ++i) {
            max = std::max(max, elems[i].key);
        }
        
        TVector<uint32_t> counters(max + 1, 0);
        for (size_t i = 0; i < elems.Size(); ++i) {
            ++counters[elems[i].key];
        }
        for (size_t i = 1; i < counters.Size(); ++i) {
            counters[i] += counters[i - 1];
        }
       
        TVector<TPair> result(elems.Size());
        for (int i = elems.Size() - 1; i >= 0; --i) {
            result[counters[elems[i].key] - 1].key = elems[i].key;
            memcpy(result[counters[elems[i].key] - 1].value, elems[i].value, sizeof(elems[i].value));
            --counters[elems[i].key];
        }
    
        elems = result;
    }
\end{lstlisting}

Обработка ввода-вывода в соответствии с описанием задания:

\begin{lstlisting}[language=C++]
    #include "pair.hpp"
    #include "vector.hpp"
    #include "sort.hpp"
    
    int main() {
        std::ios_base::sync_with_stdio(false);
        std::cin.tie(nullptr);
    
        TVector<TPair> v;
        TPair pair;
    
        while (std::cin >> pair) {
            v.PushBack(pair);
        }
    
        CountingSort(v);
        for (size_t i = 0; i < v.Size(); ++i) {
            std::cout << v[i] << '\n';
        }
    
        return 0;
    }
\end{lstlisting}

Обработка ввода и бенчмарк:

\begin{lstlisting}[language=C++]
    #include "pair.hpp"
#include "vector.hpp"
#include "sort.hpp"

#include <iostream>
#include <cstdint>
#include <chrono>
#include <algorithm>

bool operator<(const TPair& lhs, const TPair& rhs) {
    return lhs.key < rhs.key;
}

int main() {
    std::ios_base::sync_with_stdio(false);
    std::cin.tie(nullptr);
    TVector<TPair> v;
    TVector<TPair> a;
    TPair pair;

    auto start = std::chrono::steady_clock::now();
    while (std::cin >> pair) {
        v.PushBack(pair);
    }
    auto finish = std::chrono::steady_clock::now();
    auto dur = finish - start;
    std::cerr << "input" << ' ' << std::chrono::duration_cast<std::chrono::milliseconds>(dur).count() << " ms" << std::endl;

    start = std::chrono::steady_clock::now();
    std::stable_sort(v.begin(), v.end());
    //CountingSort(v);
    finish = std::chrono::steady_clock::now();
    dur = finish - start;
    std::cerr << "sort" << ' ' << std::chrono::duration_cast<std::chrono::milliseconds>(dur).count() << " ms" << std::endl;

    start = std::chrono::steady_clock::now();
    for (const auto& i : v) {
        std::cout << i << '\n';
    }
    finish = std::chrono::steady_clock::now();
    dur = finish - start;
    std::cerr << "output" << ' ' << std::chrono::duration_cast<std::chrono::milliseconds>(dur).count() << " ms" << std::endl;

    return 0;
}  
\end{lstlisting}

\pagebreak

\section{Консоль}

\begin{alltt}
    MacBook:solution vladislove$ make
    g++ -std=c++17 -pedantic -g -Wall -Wextra -Wno-unused-variable -c lab1.cpp -o lab1.o
    g++ -std=c++17 -pedantic -g -Wall -Wextra -Wno-unused-variable lab1.o -o solution
    MacBook:solution vladislove$ ./solution
    543253 abc
    000000 def
    544253 abc
    000000 zzz
    101010 mmm
    355555 zzz
    702116 xyz
    000000  def
    000000  zzz
    101010  mmm
    355555  zzz
    543253  abc
    544253  abc
    702116  xyz
\end{alltt}

\pagebreak

